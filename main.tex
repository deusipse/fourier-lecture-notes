\documentclass{amsart}

\usepackage[asy, graphs]{amogus}

\pgfplotsset{ticks=none}

\title{Lecture Notes---Fourier series}
\author{Edward Wang}

\begin{document}
  \maketitle
  \tableofcontents
  
  \section{Introduction}

  Fourier series have countless applications throughout applied mathematics and physics. Fourier series were first formally introduced by Joseph Fourier in his 1822 work \emph{Théorie analytique de la chaleur (Analytical theory of heat)} in order to solve the Heat Equation. Whilst this is not a history lesson, it is still important to know that Fourier series are rooted in physics and are commonly used to solve differential equations.

  \section{Fourier series}

  But what \emph{are} Fourier series? Despite sounding very fancy, the concept is very simple. The general concept of a Fourier series is to break a function down into a sum of sines and cosines. This is motivated by several reasons---they are easily differentiable, integrable, and manipulable.

  For example, given some $2\pi$ periodic function $f(x)$, we can express it as some sum of sinusoidal waves like so:
  \[
    f(x) \approx \frac{a_0}{2} + \sum_{m=1}^{n} (a_m \cos mx + b_m \sin mx)
  .\] If, as the number of terms $n$ approaches $\infty$, the approximation \emph{converges} to $f$, then the resulting sum is called a \emph{Fourier series}:
  \[
    f(x) = \frac{a_0}{2} + \sum_{m=1}^{\infty} (a_m \cos mx + b_m \sin mx)
  .\] You will have noticed that there are some coefficients: $a_0, a_m, b_m$. These are the \emph{Fourier coefficients}, and determine how much of each sine wave of each frequency is included. Before we determine these coefficients, we first discuss the \emph{orthogonality relations} of the sine and cosine functions.

  \section{Orthogonality of sine and cosine}

  There are three important formulae concerning sines and cosines:
  \begin{theorem}[Orthogonality of sine and cosine]
    For integers $m, n$, the following identities hold:
    \begin{align*}
      &\int_{-\pi}^{\pi} \sin mx \sin nx\, dx = \begin{cases} 0, \quad m\neq n \\ \pi, \quad m = n \neq 0 \end{cases} \\
      &\int_{-\pi}^{\pi} \cos mx \cos nx\, dx = \begin{cases} 0, \quad m\neq n \\ \pi, \quad m = n \neq 0 \end{cases} \\
      &\int_{-\pi}^{\pi} \sin mx \cos nx\, dx = 0
    \end{align*}
  \end{theorem}
  These identities may be proved using the product to sum trigonometric identities:
  \begin{align}
    \sin a \sin b &= \frac{1}{2}[\cos(a-b) - \cos(a+b)] \label{trig1}\\
    \cos a \cos b &= \frac{1}{2} [\cos (a-b) + \cos(a+b)] \label {trig2}\\
    \sin a \cos b &= \frac{1}{2}[\sin (a-b) + \sin(a+b)] \label{trig3}
  \end{align}
  \begin{proof}
    We use \ref{trig1} to prove the first formula:
    \begin{align*}
      \int_{-\pi}^{\pi} \sin mx \sin nx\, dx &= \frac{1}{2} \int_{-\pi}^{\pi} \cos(x(m-n)) - \cos(x(m+n))\, dx \\
                                             &= \frac{1}{2} \left[ \frac{\sin(x(m-n))}{m-n} - \frac{\sin(m+n)}{m+n} \right]_{-\pi}^{\pi}.
    \end{align*}
    Since $\sin k\pi = 0$ for any integer $k$, we have \[
      \int_{-\pi}^{\pi} \sin mx \sin nx \, dx = 0, \quad m\neq n
    .\] However, notice that the method only makes sense for $m\neq n$. If $m = n$, we may use the identity $\sin^2 a = \frac{1}{2}[1 - \cos(2a)]$ to obtain 
    \begin{align*}
      \int_{-\pi}^{\pi} \sin^2(mx)\, dx &= \frac{1}{2} \int_{-\pi}^{\pi} 1 - \cos(2mx)\, dx \\
                                        &= \frac{1}{2} \left[ x - \frac{\sin (2mx)}{2m} \right]_{-\pi}^{\pi} \\
                                        &= \pi
    \end{align*}
    The other two identities are left as an exercise to the reader.
  \end{proof}

  \section{Fourier Coefficients}

  Recall the Fourier series of a $2\pi$ periodic function on $[-\pi, \pi]$, $f(x)$:
  \begin{equation}
    f(x) = \frac{a_0}{2} + \sum_{m=1}^{\infty} (a_m \cos mx + b_m \sin mx).
  \end{equation}
  Let us assume that the series converges to $f$. We wish to calculate the coefficients $a_0, a_m, b_m$.


  \section{Wave Equation in One Dimension}
  
  We now look at some applications of Fourier series. First, we investigate the problem of a vibrating string.

  Consider a string of length $L$ lying on the $x$-axis, fixed at 2 ends: the origin $(0, 0)$ and $(L, 0)$. We will make some reasonable assumptions:
  \begin{itemize}
    \item The particles in the string move only up and down (there is no horizontal movement)
    \item The string is perfectly elastic
  \end{itemize}Let the function $u(x, t)$ describe the displacement of the string at time $t$.
  \begin{figure}[H]
    \centering
    \begin{tikzpicture}
      \begin{axis}[axis lines = left, width = \columnwidth, height = 5cm, ymax = 1, xmax = pi+0.3, clip = false, smooth]
        \addplot[domain = 0:pi] {0.5*sin(deg(x))};
        \draw (0, 0.9) node[right] {$u(x, t)$};
        \draw (pi+0.25, 0) node[above] {$x$};
        \draw[fill=black] (pi, 0) circle (2pt) node[below] {$(L, 0)$};
        \draw[fill=black] (0, 0) circle (2pt) node[below] {$(0, 0)$};
      \end{axis}
    \end{tikzpicture}
    \caption{A possible shape of the string at some point in time}
  \end{figure}
  First, we find the mass of the string in some interval $[a, b]$. Let  $\rho(x, t)$ describe the density of the string at  $x$ at time $t$. Focus on a tiny segment of the string at $x$.
  \begin{figure}[H]
    \begin{tikzpicture}
      \draw[very thick] (2, 1) -- (8, 3);
      \draw[dashed] (2, 1) -- (2, 0) node[below] {$x$} (8, 3) -- (8, 0) node[below] {$x + \Delta x$};
      \draw[dashed] (2, 1) -- (8, 1) node[midway, below] {$\Delta x$}; 
      \path(8, 3) -- (8, 1) node[midway, right] {$\Delta y$};
      \draw[->] (0, 0) -- (10, 0);
    \end{tikzpicture}
  \end{figure}
  Since this piece of string is so small, it approximates a straight line. Hence its length is clearly $\sqrt{\Delta x^2 + \Delta y^2} = \sqrt{1 + \big(\frac{\Delta y}{\Delta x}\big)^2}\Delta x$. Now the mass of this piece of string is simply $\rho(x, t)\sqrt{1 + \big(\frac{\Delta y}{\Delta x}\big)^2}\Delta x$. Imagine summing this $N$ times for all the tiny segments in the interval $[a, b]$. Thus an approximation of the mass of the string  in the interval $[a, b]$ is \[
    \sum_{i=1}^{N} \rho(x, t)\sqrt{1 + \Big(\frac{\Delta y_i}{\Delta x_i}\Big)^2}\Delta x_i
  .\] If these segments are equally spaced, then $\Delta x_i = \Delta x$. Our approximation becomes the true mass of the string when  $\Delta x$ approaches 0 and $N$ approaches $\infty$. Thus the mass of the string is \[
  \lim_{N \to \infty} \lim_{\Delta x \to 0} \sum_{i=1}^{N} \rho(x, t)\sqrt{1 + \Big(\frac{\Delta y_i}{\Delta x}\Big)^2}\Delta x = \int_{a}^{b} \rho(x, t)\sqrt{1 + \Big(\frac{\partial u}{\partial x}\Big)^2}\, dx
.\] For sake of brevity, we abbreviate $\frac{\partial u}{\partial x}$ to  $u_x$, and similarly, $\frac{\partial u}{\partial t}$ to $u_t$.

Let us now concentrate on another small segment of the string in the interval $[x, x + \Delta x]$. This piece of string has mass \[
  \int_{x}^{x+\Delta x} \rho(x, t)\sqrt{1 + u_x^2}\, dx  
\] which we will call $m$.
  \begin{figure}[H]
    \centering
    \begin{tikzpicture}
      \begin{axis}[clip = false, ymin = 0, ymax = 1, height = 5cm, width = \columnwidth, axis lines = left, xmin = 0, xmax = pi/2 + 0.4]
        \addplot[domain = 0.5 : pi/2-0.2, very thick] {sin(deg(x))};
        \addplot[domain = 0.3 : 0.8, <-] {cos(deg(0.5)) * (x-0.5) + sin(deg(0.5))};
        \addplot[domain = pi/2 - 0.5 : pi/2+0.1, ->] {cos(deg(pi/2 - 0.2)) * (x-pi/2 + 0.2) + sin(deg(pi/2 - 0.2))};
        \addplot[domain = 0.3:pi/2] {0};
        \filldraw (0.5, 0.479426) circle (1.5pt);
        \filldraw (0.5, 0.479426) circle (1.5pt);
        \filldraw (pi/2 - 0.2, 0.9800665) circle (1.5pt);
        \draw[dashed] (0.5, 0.479426) -- (0.5, 0) node[below] {$x$} (0.5, 0.479426) -- (0.5+0.25, 0.479426);
        \draw[dashed] (pi/2 - 0.2, 0.9800665) -- (pi/2 - 0.2, 0) node[below] {$x + \Delta x$} (pi/2 - 0.2, 0.9800665) -- (pi/2 - 0.2 + 0.25, 0.9800665);
        \draw (0.5+0.1, 0.479426) arc(0:41.27:0.1) node[midway, right, xshift = 3pt, yshift = 3pt] {$\theta_x$};
        \draw (pi/2 - 0.2 + 0.1, 0.9800665) arc(0:11.24:0.1) node[below right] {$\theta_{x + \Delta x}$};
      \end{axis}
    \end{tikzpicture}
    \caption{}
  \end{figure}
\end{document}
